\documentclass{article}
\usepackage[utf8]{inputenc}

\title{CP! - Problem Statement}
\author{Rajeev Kumar}
\date{}

\begin{document}
\maketitle
\section*{Problem}
The Joint Entrance Examination (JEE) is an engineering entrance assessment conducted for admission to various engineering colleges in India. It is constituted by two different examinations: the JEE-Main and the JEE-Advanced.\\\\
JEE-Main paper has three sections :- Physics, Chemistry, and Maths. Each section has 25 questions. For each correct answer, you get +4 marks, for each incorrect answer, you get -1 mark, and for each unattempted question, you get 0 mark.\\\\
You are given marks obtained by a student. Determine if the given marks is possible or not.
\section*{Input}
The input consists of multiple test cases. The first line contains a single integer t ($1 \leq t \leq 1000000$) — the number of test cases. The description of test cases follows.\\\\
The only line of each test case contains n ($-1000 \leq n \leq 1000$) - marks obtained by the student.
\section*{Output}
For each test case, output POSSIBLE or IMPOSSIBLE on a separate line.
\section*{Example}
input\\
1\\
0\\
output\\
POSSIBLE\\
\end{document}
